\documentclass{article}
\usepackage[utf8]{inputenc}
\usepackage{amsmath}
\usepackage{graphicx}
\usepackage{float}

\title{
Probabilistic Programming
%\\\vspace{4 mm}\underline{\Large House Lannister}
}

\author{
	Carlos Diaz
	\and
	Dan Hassin
	\and
	Charles Lehner
	\and
	Dan Viterise
}

\date{April 21, 2014}

\begin{document}

\maketitle

%\begin{abstract}
	%Your abstract goes here...
	%...
%\end{abstract}

\section{Introduction}
For many people just starting out, programming can seem like a confusing and tedious activity. It can be unclear for 
many new programmers where they should begin and what should come next while coding. Without any proper guidance, 
a novice programmer is left on their own to scour thousands of pages online to discover what should come next in 
their code. Then, once a person becomes more adept at programming, he or she may notice that they are typing out 
the same or similar lines of code within each new program. This repetitiveness slows down the user and can make 
programming a much more tiresome task than it needs to be. 
\subsection{Motivation}

% overview of markov chain
% application of markov chain to abstact syntrax tree

\section{Implementation}

\end{document}
