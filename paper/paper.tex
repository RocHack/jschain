\documentclass{article}
\usepackage[utf8]{inputenc}
\usepackage{amsmath}
\usepackage{graphicx}
\usepackage{float}

\title{
Probabilistic Programming
%\\\vspace{4 mm}\underline{\Large House Lannister}
}

\author{
	Carlos Diaz
	\and
	Dan Hassin
	\and
	Charles Lehner
	\and
	Dan Viterise
}

\date{April 21, 2014}

\begin{document}

\maketitle

%\begin{abstract}
	%Your abstract goes here...
	%...
%\end{abstract}

\section{Introduction}
For many people just starting out, programming can seem like a confusing and tedious activity. It can be unclear for 
many new programmers where they should begin and what should come next while coding. Without any proper guidance, 
a novice programmer is left on their own to scour thousands of pages online to discover what should come next in 
their code. Then, once a person becomes more adept at programming, he or she may notice that they are typing out 
the same or similar lines of code within each new program. This repetitiveness slows down the user and can make 
programming a much more tiresome task than it needs to be. These observations steered our group towards developing
a way for someone to code with more instruction and productivity. The Probabilistic Programming method allows for a user to code in JavaScript
using a web application. This web app provides a list of suggestions for the user's next line of code depending on where
the user is in their program. Not only is this helpful for new programmers that need suggestions for what to write next 
in their code, this system allows for expert programmers to write code with a great deal more ease and efficiency.

\subsection{Motivation}

% overview of markov chain
A primary motivation for undertaking this project was trying to find a new and interesting way to apply a Markov Chain.
A Markov Chain is a stochastic process that uses the current situation and a set of probabilities to determine what 
situation comes next. In the most simple case of a Markov Chain, the next action is based only upon the current situation
and all other previous actions are forgotten. Thus, this process is essentially memoryless because all past states can
be forgotten once a new state is reached. A good example of a Markov Chain is a random walk where at each integer value,
the process chooses a new value such that the two possible values are the current value plus 1 and the current value
minus 1. The transition probabilities for both of these options is .05. This is a good example of a Markov Chain because
the next value on the random walk only depends on the current value in the walk; all previous values are not needed 
to choose the next value \cite{markov}. 

% application of markov chain to abstact syntrax tree

\section{Implementation}

  \begin{thebibliography}{1}

  \bibitem{markov} Takis Konstantopoulos {\em Introductory lecture notes on
Markov Chains and Random Walks} Uppsala University, URL = http://www2.math.uu.se/~takis/L/McRw/mcrw.pdf
 
  \end{thebibliography}

\end{document}
